In this section we will investigate the expressive power of neural networks, presenting some results that motivate the use of neural networks as learning
model; we will also show how different architectures leads to different kind of computational power.
 

Maybe the most important result regarding the expressive power of neural networks it's due to Hornik et al. \cite{Hornik89} which basically states
\textit{'Multilayered feed foward networks with at least one hidden layer, using an arbitrary squashing function can approximate virtually any function
of interest to any desired degree of accuracy provided sufficiently many hidden units are available'}.

To give a more formal result we need first to define what \textit{approximate to any degree of accuracy means}, this concept is captured in definition
\ref{dens_compact}
 
\begin{defn}
 A subset S of $\mathbb{C}^n$ (continuoos functions in $\mathbb{R}^n$) is said to be \textit{uniformly dense on compacta in} $\mathbb{C}^n$ if $\forall$
 compact set $K\subset \mathbb{R}^n$ holds: $\forall \epsilon >0$, $\forall g(\cdot) \in \mathbb{C}^n$ $\exists f(\cdot) \in S$ such that 
 $\underset{x \in K}{\text{sup  }} \norm{f(x)-g(x)}<\epsilon$ 
 \label{dens_compact}
\end{defn}

Hornik result is contained in theorem \ref{universal_approx}.
\begin{thm}
 For every activation function $\sigma$, $\forall n\in \mathbb{N}$, feed foward neural
 networks with one hidden layer are a class of functions which is \textit{uniformly dense on compacta in} $\mathbb{C}^n$
\label{universal_approx}.
\end{thm}
Therem \ref{universal_approx} extends also to Borel measurable functions, please see \cite{Hornik89} for more details.


This result implies that FNN are \textit{universal approximators}, this is a strong argument for using such models in machine learning.
It's important to notice, however, that the theorem holds if we have \textit{sufficiently many} units. In praticice the number of units will bounded
by the machine capabilities and by computational time, of course greater the number of units greater will be the learning time. This will limit
the expressiveness of the network to a subset of all measurable functions.

\\\\POGGIO GIROSI?

\\\\RNNs

