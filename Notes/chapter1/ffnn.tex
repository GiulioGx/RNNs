
A feed foward neural network is an artificial neural network in which there are no cycles, that is to say each layer output is \textit{fed} to the next one and connections to any earlier layer are not possible. 


\begin{defn}[Feed foward neural network]
\label{def_ffnn}
A feed foward neural network is tuple
$$FFNN\triangleq<\vec{p},W,b,\sigma(\cdot),F(\cdot)>$$
\begin{itemize}
 \item $\vec{p} \in \mathbb{N}^U$ is the vector whose elements $p(k)$ are the number of neurons of layer $k$; $U$ is the number of layers
 \item $W\triangleq \{W^k_{p(k+1) \times p(k)}, k=1,...,U \}$ is the set of weight matrixes of each layer
 \item $b \triangleq \{\vec{b}^k \in \mathbb{R}^{p(k)}, k=1,...,U \} $ is the set of bias vectors
 \item $\sigma(\cdot): \mathbb{R}\rightarrow \mathbb{R}$ is the activation function
 \item $F(\cdot): \mathbb{R}^{p(U)}\rightarrow \mathbb{R}^{p(U)}$ is the output function
\end{itemize}
\end{defn}

\begin{remark}{}
Given a FFNN:
\begin{itemize}
 \item The number of output units is $p(U)$
 \item The number of input units is $p(1)$
 \item The total number of weights is $\mathcal{N}(W) \triangleq \sum_{k=1}^U p(k+1)p(k)$
 \item The total number of biases is $\mathcal{N}(b) \triangleq \sum_{k=1}^U p(k)$
\end{itemize}
\end{remark}

\begin{defn}[Output of a FFNN]
Given a $FFNN$ and an input vector $\vec{x} \in \mathbb{R}^{p(1)}$ the output $\vec{y} \in \mathbb{R}^U$ of the net is defined by the following:

\begin{align}
&\vec{y}=F(\vec{a}^{U}) &\\
&\vec{\phi}^{i} \triangleq \sigma(\vec{a}_{i}), & i=1,...,U\\
&\vec{a}^{i} \triangleq W^{i+1} \cdot \vec{\phi}^{i-1} +\vec{b}^i  & i=1,...,U\\
&\vec{\phi}^{0} \triangleq \vec{x} &
\end{align}
\end{defn}

\subsection{On expressivness of ffnn}
\subsection{Learning with ffnn}

As we have seen in the previous section a model sush as $FFNN$ can approximate arbitrary well any smooth function, so a natural application of feed foward neural networks is machine learning.
To model an optimization problem we first need to define a dataset $D$ as 
\begin{equation}
D\triangleq\{x^{(i)} \in \mathbb{R}^p, y^{(i)} \in \mathbb{R}^q,  i=1,...,N\}
\end{equation}
Then we need a loss function $L_D:\mathbb{R}^{\mathcal{N}(W)+\mathcal{N}(b)} \rightarrow \mathbb{R}_{\geq 0}$ over $D$ defined as
\begin{equation}
L_D(W,b)\triangleq\frac{1}{N}\sum_{i=1}^N L(\vec{x}^{(i)},\vec{y}^{(i)}) 
\end{equation}
$L(\vec{x},\vec{y}):\mathbb{R}^{p(U)} \rightarrow \mathbb{R}$ is an arbitrary loss function computed on the $i^{th}$ example. Note that $\vec{y}$ is the output of the
networks, so it depends on $(W,b)$


The problem is then fo find a $FFNN$ which minimize $L_D$. As we have seen feed foward neural network allow for large customization: the only variables in the optimization problem are the weights, the other
parameters are said \textit{hyper-parameters} and are determined \textit{a priori}. Usually the output function is choosen depending on the output, for instance for multi-way classification
is generally used the softmax function, for regression a simple identity function.

The activation function $\sigma(\cdot)$ is often choosen from:
\begin{equation}
 sigmoid(x)=\frac{1}{1-e^{-x}}
\end{equation}
\begin{equation}
 tanh(x)=\frac{e^x-e^{-x}}{e^x+e^{-x}}
\end{equation}

\begin{equation}
  ReLU(x)=\begin{cases}
    x, & \text{if $x>0$}.\\
    0, & \text{otherwise}.
  \end{cases}
\end{equation}

For what concerns the number of layers and the number of units per layers they are choosen relying on experience or performing some kind of hyper-parameter tuning, which usually consists on training nets
with some different configurations of such parameters and choosing the best one.

Once we have selected the values for all hyper-paramters the optimization problem becomes:

\begin{align}
& \underset{W,b}{\text{min}}
& & L_D(W,b) \\
\end{align}


\subsection{Gradient}
Consider a $FFNN=<\vec{p},W,b,\sigma(\cdot),F(\cdot)>$, let $L:\mathbb{R}^{p(U)} \rightarrow \mathbb{R}$ a loss function and 
$g(\cdot):\mathbb{R}^{\mathcal{N}(W)+\mathcal{N}(b)} \rightarrow \mathbb{R}$ be the function defined by
$$g(W,b) \triangleq L(F(a(W,b)))$$

\begin{equation}
\begin{align}
\frac{\partial g}{\partial \vec{w}} &= \nabla L \cdot J(F) \cdot \frac{\partial \vec{a}^U}{\partial \vec{w}}\\
&= \frac{\partial g}{\partial \vec{a}} \cdot \frac{\partial \vec{a}^U}{\partial \vec{w}}
\end{align}
\end{equation}

We can easily compute $\frac{\partial g}{\partial \vec{a}}$ once we define $F(\cdot)$ and $L(\cdot)$, note that the weights are not involved in such computation.
Let's derive an expression for $\frac{\partial \vec{a}^U}{\partial \vec{w}}$.

We will start deriving such derivative using linear notation. Let's consider a single output unit $u$ and a weight $w_lj$ linking neuron $j$ to neuron $l$.

\begin{equation}
\begin{align}
\frac{\partial a_u}{\partial w_lj} &= \frac{\partial a_u}{\partial a_l} \cdot \frac{\partial a_l}{\partial w_lj}\\
&=\delta_l^u \cdot \phi_j
\end{align}
\end{equation}

where we put $$\delta_l^u \triangleq \frac{\partial a_u}{\partial a_l}$$.

Let $P(l)$ be the set of parents of neuron $l$, formally:
\begin{equation} 
P(l) = \{ k: \exists \text{ a link between $l$ and $k$ with weight } w_{lk} \}
\end{equation}
We can compute $\delta_l^u$ simply applying the chain rule, if we write it down in bottom-up style, as can be seen in figure \ref{deriv_arcs}, we obtain:
\begin{equation}
\delta_l^u = \sum_{k\in P(l)} \delta_k^u \cdot \sigma'(a_k)\cdot w_kl
\end{equation}

\tikzstyle{rnn_style}=[->,shorten >=1pt,auto,node distance=1.5cm,
  thick,
  neuron/.style={circle,fill=white!50,draw,minimum size=0.7cm,font=\sffamily\normalsize},
  missing/.style={circle,fill=white!50,draw=none,minimum size=0.7cm,font=\sffamily\Huge\bfseries},
  label/.style={node distance=1.2cm,rectangle,fill=white!50,draw=none,minimum size=0.7cm,font=\sffamily\normalsize},
  thick_edge/.style={line width=1.2pt},
  thin_edge/.style={line width=0.5pt}
  ]
\begin{figure}[]
 \centering
\begin{tikzpicture}[rnn_style]

  \node[neuron]    (u)       {$u$};
  
  \node[neuron]    (x1)[left of=u, below of=u]   {};
  \node[neuron]    (x2)[right of=x1]   {};
  \node[neuron]    (x3)[right of=x2]   {};
  
  \node[neuron]    (y1)[below of=x1]   {$l$};
  \node[neuron]    (y2)[right of=y1]   {};
  \node[neuron]    (y3)[right of=y2]   {};
  
  \node[neuron]    (z1)[below of=y1]   {};
  \node[neuron]    (z2)[right of=z1]   {};
  \node[neuron]    (z3)[right of=z2]   {};
  
%   \node[label]      (lu)[left of=u] {$u$};
%   \node[label]      (ll)[left of=z1] {$l$};
  
  
  \path[->] (x1) edge [thick_edge] node[]{}   (u)
	    (x2) edge [thick_edge]   (u)
	    (x3) edge [thick_edge]   (u)
	    (y1) edge [thick_edge]   (x1)
	    (y1) edge [thick_edge]   (x2)
	    (y1) edge [thick_edge]   (x3)
	    (y1) edge [thin_edge]   (x2)
	    (y2) edge [thin_edge]   (x3)
	    (y3) edge [thin_edge]   (x1)
	    (y1) edge [thin_edge]   (x3)
	    (y2) edge [thin_edge]   (x2)
	    (y3) edge [thin_edge]   (x1)
	  
	    (z1) edge [thin_edge]   (y1)
	    (z1) edge [thin_edge]   (y2)
	    (z1) edge [thin_edge]   (y3)
	    (z1) edge [thin_edge]   (y2)
	    (z2) edge [thin_edge]   (y3)
	    (z3) edge [thin_edge]   (y1)
	    (z1) edge [thin_edge]   (y3)
	    (z2) edge [thin_edge]   (y2)
	    (z3) edge [thin_edge]   (y1);


\end{tikzpicture}
\caption{Nodes involved in $\frac{\partial a_u }{\partial a_l}$}
\label{deriv_arcs}
\end{figure}


In matrix notation we can rewrite the previous equations as:
\begin{equation}
 \frac{\partial a^U}{\partial \vec{w^i}} = \frac{\partial a^U}{\partial a^i} \cdot \frac{\partial a^i}{\partial \vec{w^i}}
\end{equation}

\begin{equation}
\frac{\partial a^i}{\partial \vec{w^i}} =
 \begin{bmatrix}
   \phi_1^{i-1}    & 0                & \cdots      & \cdots       & 0  \\
   0               & \phi_2^{i-1}     & \cdots      & \cdots       & 0  \\
   \vdots          & \vdots           & \ddots      & \vdots       &\vdots\\
   0               & \cdots           & \cdots      & \cdots       & \phi^{i-1}_{p(i-1)}
\end{bmatrix}
\end{equation}

\begin{equation}
\frac{\partial a^U}{\partial a^i} \triangleq \Delta^i = \Delta^{i+1} \cdot diag(\sigma'(\vec{a}^{i+i})) \cdot W^{i+1}
\end{equation}

where
\begin{equation}
diag(\sigma'(\vec{a}^{i+i})) =
 \begin{bmatrix}
   \sigma'(a^{i+1}_1)    & 0                & \cdots      & \cdots       & 0  \\
   0                     & \sigma'(a^{i+1}_2)     & \cdots      & \cdots       & 0  \\
   \vdots                & \vdots           & \ddots      & \vdots       &\vdots\\
   0                     & \cdots           & \cdots      & \cdots       &\sigma'(a^{i+1}_{p(i+1)})
\end{bmatrix}
\end{equation}







