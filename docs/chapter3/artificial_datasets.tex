\section{Artificial datasets}

The problems discussed below were first introduced by Hochreiter \cite{lstm} as difficult tasks for RNNs because they require learning long term correlations and have been used as benchmarks since. A few additions are taken from Martens\cite{hessianFree}.

\paragraph{The addition problem}
The problem consists in performing an addition between two real numbers $x_i$ and $x_j$ in $[-1,1]$ belonging to a sequence of randomly generated numbers. The difficulty in this problem is that such numbers can be arbitrarily
distant in the input sequence, so the learning net must exhibit a long term memory. More specifically the input is a sequence of pairs; each pair is composed of a real number and a marker $\in\{1,0\}$. The marker is used to select the two numbers in the sequence to add. The prediction is the last value in the output sequence, the target is $\frac{x_i+x_j}{2}$. The prediction $y$ is considered correct
if $|y-\frac{x_i+x_j}{2}| < 0.04$.

Sequences have random length, say $L$, between the minimal sequence length $T$ and $T+\frac{T}{10}$, the position of the first marker is sampled in first $\frac{L}{10}$ positions, the last marker is
instead sampled in $[\frac{4L}{10},\frac{5L}{10}]$

\paragraph{The multiplication problem}
The problem is very similar to the addition problem, here we select two numbers in the input sequences of real numbers in $[0,1]$ and we need to predict the product.

\paragraph{The XOR problem}
Again, the problem is the same as the addition one but the input are binary and we are asked to predict the XOR binary operation. This problem has been found particularly hard for both LSTM and hessian-free methods as reported in \cite{hessianFree}.

\paragraph{The temporal order problem}
The input sequences are composed of $T$ randomly chosen symbols in $\{a,b,c,d\}$ except for two randomly selected positions for which the symbols are sampled in $\{x,y\}$.
The task is to predict the relative order of the two special symbols, that is $\{xx,xy,yx,yy\}$. A variant of the task is to use three special symbols instead of two.
Again, the difficulty of the problem is the possibly distance from the special symbols whose relative order is to be detected. 
