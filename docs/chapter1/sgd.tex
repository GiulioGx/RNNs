In this section we will describe a framework based on gradient descent optimization method which can be used to train 
neural network of any kind. Such framework constitutes the core of many learning methods used in today's applications. 
Suppose we have a training set of pairs $D=\{\pair{\vec{x}^{(i)}}{\vec{y}^{(i)}}\}$ and a loss function $L(\theta)$ 
where $\theta$ represents all the parameters of the network.

A standard gradient descend would update $\theta$ at each iteration using the gradient computed on the whole training 
set, as shown below.
\begin{equation}
 \theta = \theta - \alpha \nabla_\theta L(\theta)
\end{equation}

This can be very slow or even impractical if the training set is too huge to fit in memory. Stochastic gradient descent 
overcome this problem taking into account only a part of the training set for each iteration, i.e the gradient is 
computed only on a subset $I$ of training examples. 

\begin{equation}
 \theta = \theta - \alpha \nabla_\theta L(\theta; \pair{\vec{x}^{(i)}}{\vec{y}^{(i)}}, i\in I)
 \label{eq:updateRule}
\end{equation}

The subset of training examples used for the update is called \textit{minibatch}. The number of examples for each 
minibatch is an important hyper-parameter because it affects both the speed of convergence in terms of number of 
iteration and the time needed for each iteration. At each iteration new exampled are chosen among the training set 
examples, so it could, and it always does in real applications, happen that all training set exampled have been used.
This is not a problem, since we can use the same examples over and over again. Each time we go over the entire training 
set we say we completed and \textit{epoch}. It's not unusual to iterate the learning algorithm for several epochs 
before converging.

The method is summarized in algorithm \ref{algo:sgd}.

\begin{algorithm}[]
 \KwData{\\
 \Indp
  $D=\{\pair{\vec{x}^{(i)}}{\vec{y}^{(i)}}\}$: training set\\
  $\theta_0$: initial solution \\
  $m$: size of each minibatch\\
  }
  
 \KwResult{\\
 \Indp $\theta$: solution
 }
 \BlankLine
 
 $\theta \gets \theta_0$\\
 \While{stop criterion}{
 
 $I$ $\gets$ select $m$ training example $\in D$  \\
 $\alpha \gets$ compute learning rate \\
 $\theta \gets \theta - \alpha \nabla_\theta L(\theta; \pair{\vec{x}^{(i)}}{\vec{y}^{(i)}}, i\in I)$\\
 }
\caption{Stochastic gradient descent}
\label{algo:sgd}
\end{algorithm}

In the following subsection we will analyze in more detail each step of the method, presenting different alternatives 
that can be used.

\subsection{The stop criterion}

Usually a gradient based method adopts a stop criterion which allows the procedure to stop when close enough to a (local) 
minimum, i.e $\nabla_\theta L(\theta)=0$.  This could easily lead to over-fitting, so is common practice to use a 
cross-validation technique. The most simple approach to cross-validation is to split the training set in two parts, one actually used as a pool of training examples, which will be called training set, and the other, called \textit{validation} set, used to decide when to stop.

Being $D=\{\pair{\vec{x}^{(i)}}{\vec{y}^{(i)}}, i\in(1,M)\}$ a generic subset of the dataset, we can define the \textit{error} on such set in a straightforward manner as 

\begin{equation}
 E_D = \frac{1}{M} \sum_{i=1}^M  L(\vec{x}^{(i)},\vec{y}^{(i)})
\end{equation}

Since training examples are sampled from the training set, the error on the training set will always\footnote{This is not actually true; it would in a standard gradient descent, but since we are using stochastic gradient the error could be non monotonic decreasing; however the matter here is that error mainly follow a decreasing path} be decreasing across iterations. The idea behind cross-validation is to compute, and \textit{monitor} the error on the validation set, since it's not guaranteed at all that the error would be decreasing. On the contrary, tough error will generally decrease during the first part of training, it will reach a point when it will become to increase. This is the point when we need to stop training since we are starting to over-fitting. Of course this is an ideal situation, in real applications the validation error could have a more irregular trend, but the idea holds.


DISEGNO CON LE DUE CURVE


\subsection{How to choose batches}

Beside the actual number of examples for each batch, which plays an important role in speed of convergence, it's often important how the examples are sampled.
The idea is that some examples can be more difficult than other, and so the order in which they are fed to the network matter. Of course it's not well defined what the difficulty of an example is; how can when decide if an example is more difficult than another?  We cannot possibly know, a priory, if an example will be well classified before we actually run it through the network. However we can guess the difficulty on the example by some of it's properties, for instance, if we are dealing with sequences, we can judge it's difficulty by it's length.

TODO leggere meglio curriculum learning


\subsection{Learning rate}

The parameter $\alpha$ in equation \ref{eq:updateRule}, know as \textit{step} in the optimization field, is called in AI \textit{learning rate}. Of course the strategy employed to compute such learning rate is an important ingredient in the learning method. 


\paragraph{Constant learning rate}
The most easy, and often preferred, strategy is that of \textit{constant learning rate}; Learning rate $\alpha$ becomes another hyper-parameter of the network that can be tuned, but it remains equal, usually a very small value, across all iteration.

\paragraph{Momentum}
Another popular strategy is that of \textit{momentum} which, in the optimization field is know as the \textit{Heavy Ball} method.

\begin{align}
v &= \gamma v+ \alpha \nabla_\theta L(\theta; \pair{\vec{x}^{(i)}}{\vec{y}^{(i)}}, i\in I)\\
\theta &= \theta - v
\end{align}

\paragraph{Annealing}
\paragraph{Line search}
\subsection{Regularization}
\paragraph{weight decay}
\paragraph{drop out}


