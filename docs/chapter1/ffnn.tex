
A feed forward neural network is an artificial neural network in which there are no cycles, that is to say each layer output is \textit{fed} to the 
next one and connections to earlier layers are not possible. 


\begin{defn}[Feed forward neural network]
\label{def_ffnn}
A feed forward neural network is tuple:
$$\net{FFNN}\defeq \langle\vec{p},\set{W},\set{B},\sigma(\cdot),F(\cdot)\rangle$$
\begin{itemize}
 \item $\vec{p} \in \mathbb{N}^U$ is the vector whose elements $p(k)$ are the number of neurons of layer $k$; $U$ is the number of layers
 \item $\set{W} \defeq \{W^k_{p(k+1) \times p(k)}, k=1,...,U-1 \}$ is the set of weight matrices of each layer
 \item $\set{B} \defeq \{\vec{b}^k \in \mathbb{R}^{p(k)}, k=1,...,U \} $ is the set of bias vectors
 \item $\sigma(\cdot): \mathbb{R}\rightarrow \mathbb{R}$ is the activation function
 \item $F(\cdot): \mathbb{R}^{p(U)}\rightarrow \mathbb{R}^{p(U)}$ is the output function.
\end{itemize}
\end{defn}

\begin{remark}{}
Given a $\net{FFNN}$:
\begin{itemize}
 \item The number of output units is $p(U)$
 \item The number of input units is $p(1)$
 \item The total number of weights is $\mathcal{N}(\set{W}) \defeq \sum_{k=1}^{U-1} p(k+1)p(k)$
 \item The total number of biases is $\mathcal{N}(\set{B}) \defeq \sum_{k=2}^{U} p(k)$.
\end{itemize}
\end{remark}

\begin{defn}[Output of a $\net{FFNN}$]
Given a $\net{FFNN}$ and an input vector $\vec{x} \in \mathbb{R}^{p(1)}$ the output of the net $\vec{y} \in \mathbb{R}^{p(U)}$  is defined by the following:

\begin{align}
&\vec{y}=F(\vec{a}^{U}) &\\
&\vec{h}^{i} \defeq \sigma(\vec{a}^{i}), & i=2,...,U\\
&\vec{a}^{i} \defeq W^{i-1} \cdot \vec{h}^{i-1} +\vec{b}^i  & i=2,...,U\\
&\vec{h}^{1} \defeq \vec{x}.u &
\end{align}
\end{defn}

\subsection{Learning with FFNNs}
A widespread application of neural networks is that of machine learning. In the following we will model an optimization problem which rely on $\net{FFNNs}$.
To model an optimization problem we first need to define a dataset $D$ as 
\begin{equation}
D\defeq\{\overline{\vec{x}}^{(i)} \in \mathbb{R}^p, \overline{\vec{y}}^{(i)} \in \mathbb{R}^q,  i=1,...,N\}
\end{equation}
The dataset $D$ is composed of $N$ training examples $\overline{\vec{x}}^{(i)}$, each one of them paired with a label $\overline{\vec{y}}^{(i)}$.

Then we need a loss function $L_D:\mathbb{R}^{\mathcal{N}(\set{W})+\mathcal{N}(\set{B})} \rightarrow \mathbb{R}_{\geq 0}$ over $D$ defined as
\begin{equation}
L_D(\set{W},\set{B})\defeq\frac{1}{N}\sum_{i=1}^N L(\overline{\vec{y}}^{(i)},\vec{y}^{(i)}(\set{W},\set{B})) 
\end{equation}
$L(\overline{\vec{y}},\vec{y}):\mathbb{R}^{p(U)} \times \mathbb{R}^{p(U)} \rightarrow \mathbb{R}$ is an arbitrary loss function computed on the $i^{th}$ example. Note that $\vec{y}$ is the output of the
network, so it depends on $\set{W}$ and $\set{B}$ whether $\overline{\vec{y}}$ is fixed within the dataset.


The problem is then to find a $\net{FFNN}$ which minimize $L_D$. As we have seen feed forward neural networks allow for large customization: the only variables in the optimization problem are the weights
and the biases, the other
parameters are called \textit{hyper-parameters} and are determined \textit{a priori}. Usually the output function is chosen depending on what we are trying to learn, for instance for k-way classification
is generally used the \textit{softmax} function \begin{equation}
softmax(x)_i\defeq \frac{e^{\vec{x}_i}}{\sum_{j=1}^k e^{\vec{x}_j} },
\end{equation} for regression a simple identity function.
For what concerns the number of layers and the number of units per layers they are chosen relying on experience or performing some kind of hyper-parameter tuning, which usually consists on training nets
with some different configurations of such parameters and choosing the 'best one'.

Once we have selected the values for all hyper-parameters the optimization problem becomes:

\begin{equation}
\min_{\set{W},\set{B}} L_D(\set{W},\set{B}) \\
\end{equation}


\subsection{Gradient}
\subsection{Gradient}

We can compute partial derivatives with respect to a single weight $w_{lj}$, using simply the chain rule, as 

$$\frac{\partial L}{\partial w_{lj}}=\frac{\partial L}{\partial a_l} \cdot \frac{\partial a_l}{\partial w_{lj}}=\delta_l \cdot \phi_j$$
where we put

\begin{equation}
\delta_l \triangleq \frac{\partial L}{\partial a_l}
\end{equation}



So we can easily compute $\delta_u = \frac{\partial L^{(i)}}{\partial a_u} $ for each output unit $u$ once we choose a differentiable loss function; note
that we don't need the weights for such a computation. 

Let $P(l)$ be the set of parents of neuron $l$, formally:
\begin{equation} 
P(l) = \{ k: \exists \text{ a link between $l$ and $k$ with weight } w_{lk} \}
\end{equation}

Again, simply using the chain rule, we can write, for each non output unit $l$:

\begin{equation}
\label{loss_deriv}
\delta_l = \sum_{k\in P(l)} \frac{\partial L^{(i)}}{\partial a_k} \cdot \frac{\partial a_k}{\partial a_l}= \sum_{k\in P(l)} \delta_k \cdot 
\frac{\partial a_k}{\partial \phi_l} \cdot \frac{\partial \phi_l}{\partial a_l} = \sum_{k\in P(l)} \delta_k \cdot 
w_{kl} \cdot \sigma'(a_l)
\end{equation}


For output units instead we can compute $\delta_u = \frac{\partial L^{(i)}}{\partial a_u} $ directly once we define the loss function.

For biases variables partial derivatives are simply given by:
$$\frac{\partial L}{\partial b_{l}}=\frac{\partial L}{\partial a_l} \cdot \frac{\partial a_l}{\partial w_{l}}=\delta_l \cdot 1$$




In the following we rewrite the previously derived equations in matrix notation.
Let us recall that the weight matrix for the $i^{th}$ layer is the $p(i) \times p(i-1)$ matrix whose elements $w_{l,k}$ are the weights of the arcs which link neuron $k$ from level $i-1$ to neuron $l$ from level $i$, where
$p(i)$ is the number of neuron layer $i$ is composed of.


We can rewrite equation \ref{loss_deriv} in matrix notation as:

\begin{equation}
 \frac{\partial L}{\partial W^i} = \frac{\partial L}{\partial \vec{a}^{i}} \cdot\Big(\frac{\partial \vec{a}^{i}}{\partial W^i}\Big)^T =
 \Delta^i \cdot (\vec{\phi}^{i-1})^T
\end{equation}

where
\begin{equation}
\Delta^i  \triangleq  \frac{\partial L}{\partial \vec{a}^{i}} 
\end{equation}

\begin{equation}
 \Delta^i = \big(W^{i+1}\big)^T \cdot \Delta^{i+1} \circ \sigma(\Delta^i)
\end{equation} 

\begin{equation}
 \frac{\partial L}{\partial b^i} = \frac{\partial L}{\partial \vec{a}^{i}} \cdot\Big(\frac{\partial \vec{a}^{i}}{\partial b^i}\Big)^T =
 \Delta^i \cdot Id
\end{equation} 


