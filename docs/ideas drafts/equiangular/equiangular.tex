\documentclass{article}
\usepackage[utf8]{inputenc}
\usepackage[italian]{babel}
\usepackage{graphicx}
\usepackage{cite}
\usepackage[hidelinks]{hyperref}
\usepackage{listings}
\usepackage{latexsym}
\usepackage{amsmath}
\usepackage{amsfonts,dsfont}
\usepackage{proof}
\usepackage{stmaryrd}
\usepackage{amssymb}
\usepackage{epstopdf}
\usepackage{pgf}
\usepackage{tikz}
\usepackage[noend]{algpseudocode}
\usepackage[ruled,noline,linesnumbered]{algorithm2e}
\usetikzlibrary{automata,arrows}
\let\emptyset\varnothing

\graphicspath{ {./images/} }

%%Notation:

%%vectors
\renewcommand{\vec}[1]{\boldsymbol{#1}}
%%sets
\newcommand{\set}[1]{\mathcal{#1}}
%%matrixes
\newcommand{\mat}[1]{#1}
%%norm
\newcommand{\norm}[1]{\left\Vert #1 \right\Vert}
%%defeq
\newcommand{\defeq}{\triangleq}
%%
\newcommand{\net}[1]{\mathrm{#1}}
%% pair<x,y>
\newcommand{\pair}[2]{\langle#1,#2\rangle}

\newcommand{\onevec}{\mathds{1}}

\title{On equiangular descent directions}
\author{Giulio Galvan}

\begin{document}
	\maketitle
	
\noindent
Consider the problem
\begin{equation}
\label{eq:equiangular_prob}
\begin{aligned}
& \min_{\vec{c},\vec{d}} &&&&-\vec{c} &\\
& \text{subject to} &&&& \vec{g}_i^T\vec{d} = c & i = 1, \ldots, T\\
& &&&&\vec{d}^T\vec{d} = 1&
\end{aligned}
\end{equation}
where $\norm{\vec{g}_i}=1$.
The Lagrangian is
\begin{equation}
	\mathcal{L}(c,\vec{d},\vec{\lambda},\mu) = -c +\lambda^T(G\vec{d}-c \onevec)+ \mu(\vec{d}^T\vec{d}-1),
\end{equation} where
\begin{equation}
G=\begin{bmatrix}
\vec{g}_1^T \\
\vec{g}_2^T \\
\vdots\\
\vec{g}_T^T\\
\end{bmatrix}
\end{equation}
We find the solution of \ref{eq:equiangular_prob} imposing
\begin{align}
\label{eq:lagrangian_c}
&\nabla \mathcal{L}_c = -1 - \vec{\lambda}^T \onevec =0\\
\label{eq:lagrangian_d}
&\nabla \mathcal{L}_d = G^T\lambda + 2\mu\vec{d}\ =0\\
\label{eq:lagrangian_lambda}
&\nabla \mathcal{L}_\lambda = G\vec{d} -\onevec c =0\\
\label{eq:lagrangian_mu}
&\nabla \mathcal{L}_\mu = \vec{d}^T\vec{d}-1 =0.
\end{align}
From (\ref{eq:lagrangian_d}) and \ref{eq:lagrangian_lambda} we get
\begin{equation}
	\label{eq:step1}
	GG^T\vec{\lambda} = -2\mu c \onevec,
\end{equation}hence, supposing $\vec{g}_i$'s linearly independent we write
\begin{equation}
	\lambda = (-2uc)(GG^T)^{-1} \onevec
\end{equation}
From \ref{eq:lagrangian_d} we have
\begin{equation}
\label{eq:step2}
	\vec{\lambda}^T GG^T \vec{\lambda} = 4\mu^2,
\end{equation} which, combined with (\ref{eq:step1}) and (\ref{eq:lagrangian_c}) yields
\begin{equation}
	c = 2\mu
\end{equation}
Again from (\ref{eq:lagrangian_c}) and (\ref{eq:step2}) we get
\begin{equation}
	\onevec^T\lambda = -2uc \onevec(GG^T)^{-1}\onevec,
\end{equation} hence 
\begin{equation}
	\mu \pm \frac{1}{2\sqrt{\onevec(GG^T)^{-1}\onevec}}.
\end{equation}
Since we are interested in a positive value for $c$ the solution is
\begin{align}
	& c = 2\mu\\
	&\vec{\lambda} = (-4\mu^2)(GG^T)^{-1} \onevec\\
	&\vec{d} = -\frac{G^T\vec{\lambda}}{2\mu}\\
	&\mu = \frac{1}{2\sqrt{\onevec(GG^T)^{-1}\onevec}}
\end{align}


\end{document}      



