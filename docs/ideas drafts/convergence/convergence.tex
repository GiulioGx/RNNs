 
\documentclass{article}
\usepackage[utf8]{inputenc}
\usepackage[italian]{babel}
\usepackage{graphicx}
\usepackage{cite}
\usepackage[hidelinks]{hyperref}
\usepackage{listings}
\usepackage{latexsym}
\usepackage{amsmath}
\usepackage{proof}
\usepackage{stmaryrd}
\usepackage{amssymb}
\usepackage{epstopdf}
\usepackage{pgf}
\usepackage{tikz}
\usepackage[noend]{algpseudocode}
\usepackage[ruled,noline,linesnumbered]{algorithm2e}
\usetikzlibrary{automata,arrows}
\let\emptyset\varnothing

\graphicspath{ {./images/} }

%%Notation:

%%vectors
\renewcommand{\vec}[1]{\boldsymbol{#1}}
%%sets
\newcommand{\set}[1]{\mathcal{#1}}
%%matrixes
\newcommand{\mat}[1]{#1}
%%norm
\newcommand{\norm}[1]{\left\Vert #1 \right\Vert}
%%defeq
\newcommand{\defeq}{\triangleq}
%%
\newcommand{\net}[1]{\mathrm{#1}}
%% pair<x,y>
\newcommand{\pair}[2]{\langle#1,#2\rangle}

\title{Stochastic gradient descent}
\author{Giulio Galvan}

\begin{document}
	\maketitle
\noindent
Consider the stochastic optimization problem 
\begin{equation}
	\underset{\vec{x} \in \mathbb{R}^n}{\text{min}} f(\vec{x}) = \mathbb{E[F(\vec{x}, \vec{\xi})]},
	\label{eq:stochastic_prob}
\end{equation}
where $\vec{\xi} \in \Omega \subset \mathbb{R}^d$ is a random vector.
Suppose $f(\cdot)$ is continuous, strongly convex and there exists a compact level set of $f(\cdot)$, hence (\ref{eq:stochastic_prob}) has a unique optimal solution $\vec{x}_*$. Let also $L$ be the Lipschitz constant of $\nabla f$.
We make the following two assumptions:
\begin{itemize}
	\item	It is possible to generate independent identically distributed samples of $\vec{\xi}$
	\item There exists an oracle which, for a given point $(\vec{x}, \vec{\xi})$ return a stochastic descent direction $D(\vec{x}, \vec{\xi})$ such that $d(\vec{x})\defeq\mathbb{E}[D(\vec{x}, \vec{\xi})]$ satisfy:
	\begin{equation}
	-(\vec{x}-\vec{x}_*)^T (\nabla f(\vec{x}) -g(\vec{x})) \geq -\mu L \norm{\vec{x}_j-\vec{x}_*}^2_2\quad \forall \vec{x} \in \mathbb{R}^n,
	\label{eq:angular_condition}
	\end{equation}
	for some $\mu \in (0,1) $.
\end{itemize}
Consider the algorithm defined by
\begin{equation}
	\vec{x}_{j+1} = \vec{x}_j -\gamma_j D(\vec{x}_j,\vec{\xi}_j).
	\label{eq:stochastic_algo}
\end{equation}
Let $A_j\defeq \norm{\vec{x}_j-\vec{x}_*}^2_2$ and $a_j\defeq\mathbb{E}[A_j]$.
From (\ref*{eq:stochastic_algo}) we get
\begin{equation}
\begin{aligned}
	A_{j+1} &= \frac{1}{2}\norm{\vec{x}_j - \gamma_jD(\vec{x}_j,\vec{\xi}_j) -\vec{x}_*}^2_2\\ 
	&= A_j +\frac{1}{2}\gamma_j^2\norm{D(\vec{x}_j,\vec{\xi}_j)}^2_2 - \gamma_j(\vec{x}_j-\vec{x}_*)^TD(\vec{x}_j,\vec{\xi}_j).
\end{aligned}
\label{eq:aj_rec}
\end{equation}
Since $\vec{x}_j = \vec{x}_j(\vec{\xi_{[j-1]}})$ is independent of $\vec{\xi}_j$ we have
\begin{equation}
\begin{aligned}
	\mathbb{E}[(\vec{x}_j-\vec{x}_*)^TD(\vec{x}_j,\vec{\xi}_j)] &= \mathbb{E}_{\vec{\xi}_{[j-1]}}[\mathbb{E}_{\vec{\xi}_j}[(\vec{x}_j-\vec{x}_*)^TD(\vec{x}_j,\vec{\xi}_j)]|\vec{\xi}_{[j-1]}]\\
	&= \mathbb{E}_{\vec{\xi}_{[j-1]}}[(\vec{x}_j-\vec{x}_*)^T\mathbb{E}_{\vec{\xi}_j}[D(\vec{x}_j,\vec{\xi}_j)]|\vec{\xi}_{[j-1]}]\\
	&=\mathbb{E}_{\vec{\xi}_{[j-1]}}[(\vec{x}_j-\vec{x}_*)^Td(\vec{x}_j)]\\
\end{aligned}
\label{eq:independece}
\end{equation}
Let now assume that there exists $M>0$ such that
\begin{equation}
	\mathbb{E}[D(\vec{x},\vec{\xi})]\leq M^2 \quad \forall \vec{x} \in \mathbb{R}^n.
	\label{eq:gradient_bound}
\end{equation}
Using (\ref{eq:independece}) and \ref{eq:gradient_bound} we obtain, taking expectation of both sides of (\ref{eq:aj_rec})
\begin{equation}
	a_{j+1} \leq a_j - \gamma_j\mathbb{E}_{\vec{\xi}_{[j-1]}}[(\vec{x}_j-\vec{x}_*)^Td(\vec{x}_j)] + \frac{1}{2}\gamma_j^2M^2
	\label{eq:aj_rec_2}
\end{equation}
Since $f(\cdot)$ is strongly convex there exists $c>0$ such that
\begin{equation}
	(\vec{y}-\vec{x})^T(\nabla f(\vec{y})- \nabla f(\vec{x}))\geq c \norm{\vec{y}-\vec{x}}^2_2
	\label{eq:strong_convexity}
\end{equation}
By optimality of $\vec{x}_*$ we have
\begin{equation}
	(\vec{x}-\vec{x}_*)^T\nabla f(\vec{x}_*) \geq 0 \quad \vec{x} \in \mathbb{R}^n.
	\label{eq:optimality}
\end{equation}
Inequalities (\ref{eq:optimality}) and (\ref{eq:strong_convexity}) implies
\begin{equation}
	(\vec{x}-\vec{x}_*)^T \nabla f(\vec{x}) \geq c \norm{\vec{x}-\vec{x}_*}^2_2 \quad \vec{x} \in \mathbb{R}^n.
\end{equation}
Adding and subtracting the descent direction $g(\vec{x})$ we get
\begin{equation}
	(\vec{x}-\vec{x}_*)^T (\nabla f(\vec{x}) -g(\vec{x}) +g(\vec{x})) \geq c \norm{\vec{x}-\vec{x}_*}^2_2,
\end{equation}
which can be rewritten as
\begin{equation}
		(\vec{x}-\vec{x}_*)^T g(\vec{x}) \geq c \norm{\vec{x}-\vec{x}_*}^2_2 - (\vec{x}-\vec{x}_*)^T (\nabla f(\vec{x}) -g(\vec{x}))
		\label{eq:angular_inequality}
\end{equation}
From (\ref{eq:angular_condition}), taking expectations of both side of (\ref{eq:angular_inequality}) we obtain
\begin{align}
\mathbb{E}[(\vec{x}_j-\vec{x}_*)^T g(\vec{x}_j)] &\geq c \mathbb{E}[\norm{\vec{x}_j-\vec{x}_*}^2_2)] - \mathbb{E}[(\vec{x}_j-\vec{x}_*)^T (\nabla f(\vec{x}_j) -g(\vec{x}_j))]\\
 &\geq c(1-\frac{\mu L}{c}) \mathbb{E}[\norm{\vec{x}_j-\vec{x}_*}^2_2)]\\
 & = 2\bar{c}a_j,
\end{align}
with $\bar{c}=c(1-\frac{\mu L}{c})$.
Hence from (\ref{eq:aj_rec_2}) follows 
\begin{equation}
	a_{j+1} \leq (1-2\bar{c}\gamma_j)a_j + \frac{1}{2}\gamma_j2M^2.
\end{equation}

Choosing the stepsizes as $\gamma_j = \frac{\beta}{j}$ for some constant $\beta>\frac{1}{2\bar{c}}$ we get
\begin{equation}
		a_{j+1} \leq (1-2\bar{c}\gamma_j)a_j + \frac{1}{2}\frac{\beta^22M^2}{j^2}.
\end{equation}
It follows by induction that
\begin{equation}
	\mathbb{E}[\norm{\vec{x}_j - \vec{x}_*}^2_2] = 2a_j\leq \frac{Q(\beta)}{j},
\end{equation}
where 
\begin{equation}
	Q(\beta) = max\left\{\frac{\beta^2M^2}{2\bar{c}-1},\norm{\vec{x}_1 - \vec{x}_*}^2_2 \right\}.
\end{equation}
Hence, since
\begin{equation}
	f(\vec{x})\leq f(\vec{x}_*) + \frac{1}{2}L\norm{\vec{x} - \vec{x}_*}^2_2, \quad \forall \vec{x} \in \mathbb{R}^n,
\end{equation}
we obtain
\begin{equation}
	\mathbb{E}[f(\vec{x}_j)-f(\vec{x}_*)] \leq \frac{1}{2} L \mathbb{E}[\norm{\vec{x}_j - \vec{x}_*}^2_2] \leq \frac{1}{2}LQ(\beta)
\end{equation}

\paragraph{Descent direction condition} Condition \ref{eq:angular_condition} can be further elaborated.
Let $\cos\theta_j$ be the angle between $\nabla f(\vec{x}_j)$ and $g(\vec{x}_j)$ and $\norm{g(\vec{x}_j)} = \alpha_j \norm{\nabla f(\vec{x}_j)}$ for some $\alpha_j>0$.
Then,
\begin{align}
	\norm{\nabla f(\vec{x}_j)-g(\vec{x}_j)}^2 &= \norm{\nabla f(\vec{x}_j)}^2 + \norm{g(\vec{x}_j)}^2 -2\norm{\nabla f(\vec{x}_j)}\norm{g(\vec{x}_j)}\cos\theta_j\\
	&=  \norm{\nabla f(\vec{x}_j)}^2(1+\alpha_j^2-2\alpha_j \cos\theta_j).
\end{align}
Since $\nabla f(\vec{x}_*)=0$, using Lipschitz continuity of $\nabla f$ (with constant L) we get
\begin{equation}
	\norm{\nabla f(\vec{x}_j)-g(\vec{x}_j)} \leq L\norm{\vec{x}_j-\vec{x}_*} (1+\alpha_j^2-2\alpha_j \cos\theta_j)^{\frac{1}{2}}
\end{equation}
Hence
\begin{align}
(\vec{x}_j-\vec{x}_*)^T (\nabla f(\vec{x}_j) -g(\vec{x}_j)) &\leq \norm{\vec{x}_j-\vec{x}_*} \norm{\nabla f(\vec{x}_j) -g(\vec{x}_j)}\\
&\leq L\norm{\vec{x}_j-\vec{x}_*}^2 (1+\alpha_j^2-2\alpha_j \cos\theta_j)^{\frac{1}{2}}.
\end{align}
Hence condition \ref{angular_condition} becomes
\begin{equation}
	(1+\alpha_j^2-2\alpha_j \cos\theta_j)^{\frac{1}{2}} \geq \mu
\end{equation}

\end{document}      


