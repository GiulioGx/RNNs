
Hochreiter\cite{lstm} in 1991, Bengio et al.\cite{learningIsDifficult} in 1994, and others, observed that gradient in 
deep neural networks tends to either vanish or explode. From then onward several methods have been proposed to 
overcome what is now know as the \textit{exploding/vanishing gradient} problem. We can roughly partition such methods in 
two broad categories.
The approaches of the first kind, the architectural based ones, act on the network topology, modifying the way the 
neural units operates, the connections between them or the relationship between layers; the idea of such methods is to 
build networks architectures in which gradient are less likely to vanish, or in other words whose units are able to 
store information for several time steps.

The second approach, which we'll call \textit{learning based}, instead, focus on the learning algorithm, leaving the 
network architecture untouched.
\\DESCRIZIONE MELIORE


In the rest of the chapter we will review the most relevant methods in both the approaches.
