\subsection{Structurally constrained recurrent network}

In 2015 Mikolov proposed a novel network architecture to deal with vanishing gradients \cite{scrn} called 
\textit{Structurally constrained recurrent network} (SCRN). The idea is to introduce a hidden layer specifically 
designed to capture long-term dependencies alongside the traditional one as shown in figure \ref{fig:scrn}.


\tikzstyle{rnn_style}=[->,shorten >=1pt,auto,node distance=1.5cm,
  thick,
  neuron/.style={circle,fill=white!50,draw,node distance = 1cm, minimum size=0.7cm,font=\sffamily\Large\bfseries},
  gate/.style={circle,fill=white!50,draw,node distance = 1cm,font=\sffamily\small\bfseries},
  missing/.style={rectangle,fill=white!50,node distance =1cm,draw=none,minimum size=0.7cm,font=\sffamily\Huge\bfseries},
  label/.style={node distance=1.2cm,rectangle,fill=white!50,draw=none,minimum size=0.7cm,font=\sffamily\normalsize},
  layer/.style={rectangle,fill=white!50,draw,minimum width=3.5cm,minimum height=0.5cm, font=\sffamily\normalsize},]
\begin{figure}[!h]
 \centering
\begin{tikzpicture}[rnn_style]
  
  \node[layer] (x)[] {input layer};
  \node[layer] (h)[above left =1.2cm and -1.5cm of x] {hidden layer};
  \node[layer] (s)[above right =1.2cm and -1.5cm of x] {context layer};
  \node[layer] (y)[above right = 1.2cm and -1.5cm of h,] {output layer};
  
  \node[label] (xLabel) [below of=x, node distance=1.2cm]{$\vec{x}$};
  \node[label] (yLabel) [above of=y, node distance=1.2cm]{$\vec{y}$};

  
    \path[->] (x) edge 	[] node[]{}   	(h)
	    (x) edge 	[]   		(s)
	    (h) edge	[]		(s)
	    (h) edge[]			(y)
	    (s) edge[]			(y)
	    (xLabel)edge[]		(x)
	    (y)edge[]		(yLabel)
	    (h.north)edge[bend right=120, distance = 3.5cm]	(h.south)
	    (s.north)edge[bend left=120, distance = 3.5cm]	(s.south);
	    


\end{tikzpicture}
\caption{SCRN architecture}
\label{fig:scrn}
\end{figure}


As observed in \cite{scrn}, and explained in section \ref{sec:vanishing}, gradient can vanish either because of the 
non linearities being all close to 0 or because of the multiplication of the weight matrix at each time step. The new 
layer, called \textit{context layer}, address these problem by completely removing the non linearity and forcing the 
recurrent matrix to be close to the identity. Formally the context layer  $\vec{s}$ is given by:

\begin{equation}
 \vec{s}_t = (1-\alpha)\mat{B}\vec{x}_t + \alpha\vec{s}_{t-1}
\end{equation}

The rest of the network is like a traditional one, hence, adding the context layer beside the traditional one results 
in:
\begin{equation}
 \vec{h}_t = \sigma(\mat{P}\vec{s}_t + \mat{A}\vec{x}_t + \mat{R}\vec{h}_{t-1})
\end{equation}
\begin{equation}
 \vec{y}_t = f(\mat{U}\vec{h}_t + \mat{V}\vec{s}_t)
\end{equation}
Notice the similarity with leaky integrator units \cite{leakyIntegratorUnits}.

If treat context and traditional layers as one, i.e we don't distinguish between context and traditional units, we can 
see the model as a traditional model whose recurrent matrix $\mat{W}$ is constrained (from this the name of the 
method), to be of the form:
\begin{equation}
 \mat{W} =  \begin{bmatrix}
R & P \\
0 & \alpha I
\end{bmatrix}
\end{equation}
Matrix $\mat{W}$ is a traditional recurrent matrix constrained to have a diagonal block to be equal to a weighted 
identity. 

Observing that fixing $\alpha$ to be constant makes the context units to work on the same time scale, Mikolov propose 
to have a different value for each unit, hence allowing to capture context from different time delay.
\begin{equation}
  \vec{s}_t = (\mat{I}-\mat{Q})\mat{B}\vec{x}_t + \mat{Q}\vec{s}_{t-1}
\end{equation}
where $\mat{Q}\defeq diag(\sigma(\vec{\beta}))$; the vector $\vec{\beta}$ is learned.


In \cite{scrn} SCRNs are shown to be roughly equivalent to the much more complex, LSTMs.


