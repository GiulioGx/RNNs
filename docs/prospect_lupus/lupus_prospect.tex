 
\documentclass{article}
\usepackage[utf8]{inputenc}
\usepackage[italian]{babel}
\usepackage{graphicx}
\usepackage{cite}
\usepackage[hidelinks]{hyperref}
\usepackage{listings}
\usepackage{latexsym}
\usepackage{amsmath}
\usepackage{proof}
\usepackage{stmaryrd}
\usepackage{amssymb}
\usepackage{epstopdf}
\usepackage{pgf}
\usepackage{tikz}

\usetikzlibrary{positioning, arrows,calc,shapes,decorations.pathreplacing}

\usepackage[noend]{algpseudocode}
\usepackage[ruled,noline,linesnumbered]{algorithm2e}
\let\emptyset\varnothing

\graphicspath{ {./images/} }

%%Notation:

%%vectors
\renewcommand{\vec}[1]{\boldsymbol{#1}}
%%sets
\newcommand{\set}[1]{\mathcal{#1}}
%%matrixes
\newcommand{\mat}[1]{#1}
%%norm
\newcommand{\norm}[1]{\left\Vert #1 \right\Vert}
%%defeq
\newcommand{\defeq}{\triangleq}
%%
\newcommand{\net}[1]{\mathrm{#1}}
%% pair<x,y>
\newcommand{\pair}[2]{\langle#1,#2\rangle}

\title{Lupus prospect}
\author{Giulio Galvan}

\begin{document}
\section{Experiment description}

We trained the model choosing the training set in different ways, namely filtering the data with different criteria. The goal is to predict whether a patience will result positive or not in the near (to be defined later) future, given a number of visits in which the patient results unaffected by the disease (more precisely has zero SDI).

\paragraph{Positives.} We used as positive examples all the patients which are negative at the first visit but results positive in later visits (i.e. we exclude completely all the patients which are positive from the first visit). We use as training sequences only the visits in which the patients has zero SDI .

\paragraph{Negatives.} For the negative examples we choose only the patients which satisfy some temporal constraints. The first requirement is for the recorded history of a patience to be long enough to be able to leave out the last part of it from the training sequence. This ensures that we train the model to give a prediction valid, at least, for such given span of years. We measure this time (in years) with the parameter \texttt{upper span age}.  We than require the remaining part of the visits (which are the only ones used in the training sequence) to cover a sufficiently long period of time with the parameter (in years) \texttt{lower span age} and to be composed by at least \texttt{min visits} number of visits.

\tikzstyle{rnn_style}=[shorten >=1pt,auto,node distance=0.5cm,
thick,
sdi/.style={rectangle,fill=white!50,node distance=0.7cm, draw=none,minimum size=0.5cm,font=\sffamily\normalsize},
missing/.style={circle,fill=white!50,draw,minimum size=0.7cm,font=\sffamily\Huge\bfseries},
label/.style={node distance=0.9cm and 4cm,rectangle,fill=white!50,draw=none,minimum size=0.7cm,font=\sffamily\normalsize},
thick_edge/.style={line width=1.2pt},
thin_edge/.style={line width=0.5pt}
]
\begin{figure}[h]
	\centering
	\begin{tikzpicture}[rnn_style]
	
	%SDI column
	
	\node[sdi]    (x1)[]   {$0$};
	\node[sdi]    (x2)[above of=x1]   {$0$};
	\node[sdi]    (x3)[above of=x2]   {$0$};
	\node[sdi]    (x4)[above of=x3]   {$0$};
	\node[sdi]    (x5)[above of=x4]   {$0$};
	\node[sdi]    (x6)[above of=x5]   {$0$};
	\node[sdi]    (x7)[above of=x6]   {$0$};
	\node[sdi]    (x8)[above of=x7]   {$0$};
	\node[label]  (sdiLabel)[above of = x8] {SDI};
	
	%AGE column
	
	\node[sdi]    (y1)[right =0.5cm of x1]   {$20$};
	\node[sdi]    (y2)[above of=y1]   {$21$};
	\node[sdi]    (y3)[above of=y2]   {$22$};
	\node[sdi]    (y4)[above of=y3]   {$25$};
	\node[sdi]    (y5)[above of=y4]   {$30$};
	\node[sdi]    (y6)[above of=y5]   {$35$};
	\node[sdi]    (y7)[above of=y6]   {$40$};
	\node[sdi]    (y8)[above of=y7]   {$45$};
	\node[label]  (sdiLabel)[above of = y8] {AGE};
	
	\node[label] (arr) [left = 0.7cm of x5] {last training visit};
	
	 \draw[decorate,decoration={brace,raise=8pt,amplitude=6pt}, thick]
	 (x1.center)--(x5.center) node [black,midway,xshift=-0.7cm, align=center, text width = 3cm] {
	 	training visits (at least \texttt{min visits})};
	 
	 \draw[decorate,decoration={brace,raise=8pt,amplitude=6pt, mirror}, thick]
	 	 (y5.center)--(y8.center) node [black,midway,xshift=3.5cm] {
	 	 	\texttt{upper span age}};
	 	 
	 \draw[decorate,decoration={brace,raise=8pt,amplitude=6pt, mirror}, thick]
	 		 (y1.center)--(y5.center) node [black,midway,xshift=3.5cm] {
	 		 	\texttt{lower span age}};
	 		 
	 \path[->]
	   (arr) edge []  node[]{} (x5);
	

	\end{tikzpicture}
	\caption{Example of a negative training sequence.}
	\label{fig:lupus_neg_example}
\end{figure}

\paragraph{Experiment setup}
We explored all the combination for \texttt{upper span age} in [1, 2], \texttt{lower span age} in [1, 2], \texttt{min visits} in [2, 3, 4, 5]. The results are shown in Table (\ref{table:exp_res}).


\begin{table}[!h]
	\centering
	\begin{tabular}{c  c  c | c  c c}
		upper span age & lower span age & min visits & auc roc score & pos & neg\\
		1 & 1 & 2  & 0.62 & 43 & 127\\
		1 & 1 & 3  & 0.71 & 43 & 118\\
		1 & 1 & 4  & 0.74 & 43 & 107\\
		1 & 1 & 5  & \textbf{0.77} & 43 & 87\\
		2 & 1 & 2  & 0.65 & 43 & 87\\
		2 & 1 & 3  & 0.70 & 43 & 83\\
		2 & 1 & 4  & 0.75 & 43 & 67\\
		2 & 1 & 5  & 0.75 & 43 & 48\\
		1 & 2 & 2  & 0.71 & 43 & 86\\
		1 & 2 & 3  & 0.69 & 43 & 85\\
		1 & 2 & 4  & 0.73 & 43 & 84\\
		1 & 2 & 5  & 0.76 & 43 & 74\\
		2 & 2 & 2  & 0.67 & 43 & 44\\
		2 & 2 & 3  & 0.71 & 43 & 43\\
		2 & 2 & 4  & 0.74 & 43 & 41\\
		2 & 2 & 5  & 0.75 & 43 & 35\\

	\end{tabular}
	\caption{AUC ROC score for different training sets. Best score in bold.}
	\label{table:exp_res}
\end{table}


\begin{figure}[h]
	\centering
	\includegraphics[width= 0.8\textwidth]{roc.png}
	\caption{roc curve for the best model}
	\label{fig:roc_best}
\end{figure}

\begin{figure}[h]
	\centering
	\includegraphics[width= 0.8\textwidth]{sensibility_specificity.png}
	\caption{sensibility-specificity curve for the best model}
	\label{fig:sensibility_specificity_best}
\end{figure}

\begin{figure}[h]
	\centering
	\includegraphics[width= 0.8\textwidth]{precision_recall.png}
	\caption{precision-recall curve for the best model}
	\label{fig:precision_recall_best}
\end{figure}


\end{document}      


